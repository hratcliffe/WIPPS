\documentclass[]{article}
\usepackage[lmargin = 2cm, rmargin = 2cm]{geometry}
\usepackage{amsmath}
\title{Derivation of resonant frequencies}

\begin{document}

\section{Resonant frequency}
Start from Stix Equation 2.45 for the Right hand mode, noting that $\Omega_c$ contains the \emph{sign} of the particle charge:
\begin{equation}\label{A}
\mu^2 =  1 - \frac {\omega_{pe}^2}{\omega(\omega + \Omega_e \cos\theta)}
\end{equation}
along with the resonance condition
\begin{equation}
\omega -k_\parallel v_\parallel = \frac{n \Omega_e}{\gamma}
\end{equation}
The latter is rewritten as
\begin{equation}
k_\parallel = k \cos\theta = \frac{1}{v_\parallel} \left(\omega - \frac{n \Omega_e}{\gamma} \right) \end{equation}
and so as $ \mu = c k /\omega = (c k_\parallel)/( \omega \cos\theta)$ we get
\begin{equation}\label{B}
\mu = \frac{c ( \gamma \omega - n \Omega_e)}{\gamma\omega v_\parallel \cos\theta}.
\end{equation}
Now we equate \ref{A} and \ref{B}:
\begin{equation}
\frac{\omega(\omega + \Omega_e \cos\theta) - \omega_{pe}^2}{\omega(\omega + \Omega_e \cos\theta)} = \frac{c^2}{\gamma^2 \omega^2 v_\parallel^2 \cos^2\theta} (\gamma \omega - n \Omega_e)^2 \end{equation}
which rearranges to
\begin{equation}
(\gamma^2 \omega^2 v_\parallel^2 \cos^2\theta) (\omega(\omega + \Omega_e \cos\theta) - \omega_{pe}^2) = c^2 \omega(\omega + \Omega_e \cos\theta)(\gamma \omega - n \Omega_e)^2. \end{equation}
Cancel one $\omega$ from each side:
\begin{equation}
(\gamma^2 \omega v_\parallel^2 \cos^2\theta) (\omega(\omega + \Omega_e \cos\theta) - \omega_{pe}^2) = c^2 (\omega + \Omega_e \cos\theta)(\gamma \omega - n \Omega_e)^2 \end{equation}
and multiply out:
\begin{equation}
(\gamma^2 \omega v_\parallel^2 \cos^2\theta) (\omega(\omega + \Omega_e \cos\theta) - \omega_{pe}^2) = c^2 (\omega + \Omega_e \cos\theta)(\gamma^2 \omega^2 -2 n \Omega_e \gamma \omega + n^2 \Omega_e^2) \end{equation}
and again
\begin{align}
\gamma^2\omega^3 v_\parallel^2 \cos^2 \theta + \gamma^2 \omega^2 v_\parallel^2 \cos^3\theta\Omega_e - \gamma^2\omega v_\parallel^2\cos^2\theta\omega_{pe}^2 &=\notag\\ c^2\gamma^2 \omega^3 + c^2 \Omega_e\cos\theta \gamma^2\omega^2 - 2 c^2 \omega^2 n\Omega_e\gamma - 2 c^2 n \Omega_e^2 \gamma\omega\cos\theta + c^2 \omega n^2\Omega_e^2 + c^2 n^2\Omega_e^3\cos\theta &
\end{align}

Write $v_\parallel / c = \tilde{v}$ and divide through by $c$:
\begin{align}
\gamma^2\omega^3 \tilde{v}^2 \cos^2 \theta + \gamma^2 \omega^2 \tilde{v}^2 \cos^3\theta\Omega_e - \gamma^2\omega \tilde{v}^2\cos^2\theta\omega_{pe}^2 &=\notag\\ \gamma^2 \omega^3 +  \Omega_e\cos\theta \gamma^2\omega^2 - 2  \omega^2 n\Omega_e\gamma - 2  n \Omega_e^2 \gamma\omega\cos\theta +  \omega n^2\Omega_e^2 +  n^2\Omega_e^3\cos\theta&
\end{align}
Now collect up coefficients of each power:
\begin{align}
\left(\gamma^2 \tilde{v}^2 \cos^2 \theta - \gamma^2   \right)&\omega^3  \notag\\
+\left(\gamma^2 \tilde{v}^2 \cos^3\theta\Omega_e  -\Omega_e\cos\theta \gamma^2  + 2 n\Omega_e\gamma  \right)&\omega^2  \notag\\
+\left(- \gamma^2 \tilde{v}^2\cos^2\theta\omega_{pe}^2  +2  n \Omega_e^2 \gamma\cos\theta  - n^2\Omega_e^2   \right)&\omega \notag\\
+\left(  -n^2\Omega_e^3\cos\theta   \right)&
\end{align}
Or as $\omega , \omega_{pe}, |\Omega_e| \gg 1$ we can make the substitution $ \omega_0 = \omega/\omega_{ref}$ to minimise the differences in size between the coefficients, giving:
\begin{align}
\left(\gamma^2 \tilde{v}^2 \cos^2 \theta - \gamma^2   \right)&(\omega_{ref}\omega_0)^3 \notag \\
+\left(\gamma^2 \tilde{v}^2 \cos^3\theta\Omega_e  -\Omega_e\cos\theta \gamma^2  + 2 n\Omega_e\gamma  \right)&(\omega_{ref}\omega_0)^2  \notag\\
+\left(- \gamma^2 \tilde{v}^2\cos^2\theta\omega_{pe}^2  +2  n \Omega_e^2 \gamma\cos\theta  - n^2\Omega_e^2   \right)&\phantom{(}\omega_{ref}\omega_0\phantom{)^2}  \notag\\
+\left(  -n^2\Omega_e^3\cos\theta   \right)&\phantom{(\omega_ref\omega_0)^2} 
\end{align}
or dividing through by $(\omega_{ref})^3$:
\begin{align}
\left(\gamma^2 \tilde{v}^2 \cos^2 \theta - \gamma^2   \right)&(\omega_0)^3 \notag \\
+\left(\gamma^2 \tilde{v}^2 \cos^3\theta(\Omega_e/\omega_{ref})  -(\Omega_e/\omega_{ref})\cos\theta \gamma^2  + 2 n(\Omega_e/\omega_{ref})\gamma  \right)&(\omega_0)^2  \notag\\
+\left(- \gamma^2 \tilde{v}^2\cos^2\theta(\omega_{pe}/\omega_{ref})^2  +2  n (\Omega_e/\omega_{ref})^2 \gamma\cos\theta  - n^2(\Omega_e/\omega_{ref})^2   \right)&\phantom{(}\omega_0\phantom{)^2}  \notag\\
+\left(  -n^2(\Omega_e/\omega_{ref})^3\cos\theta   \right)&\phantom{(\omega_ref\omega_0)^2} 
\end{align}
which are expected to be order 1 for $n \sim 1$ and $\tilde{v}, \cos\theta \sim 1$.

\section{Normalisation of angular distribution}
\newcommand{\atan}{\mathrm{atan}}
\newcommand{\st}{\right|}

We need to calculate the factor $N(\omega)$ in Lyons 1974  J Plasma Phys, 12, 417, Equation A3 or A5 for arbitrary $\mu$. I.e. 
\begin{equation}
N(\omega) = \frac{2}{(2\pi)^2} \int_0^\infty g_\omega( \tan^{-1} x ) \left[ J \left( \frac{k_\perp, k_\parallel}{\omega, x} \right)\right] k_\perp d x
\end{equation}
Using Eq A6:
\begin{equation}
J \left( \frac{k_\perp, k_\parallel}{\omega, x} \right) = - \left.\frac{\partial k_\perp}{\partial x}\right|_\omega \cdot \left.\frac{\partial k_\parallel}{\partial \omega}\right|_{k_\perp}, 
\end{equation}
$k_\perp = x k_\parallel$ and $ \mu = c k /\omega$ we get 
\begin{equation}
\frac{\partial k_\perp}{\partial x} = \left. \frac{\partial}{\partial x} \left( x \cos(\atan x) \frac{\mu(x) \omega}{c}\right)\st_\omega.
\end{equation}
Using the chain rule:
\begin{equation}
\left.\frac{\partial k_\perp}{\partial x}\st_\omega = \left. \frac{\partial}{\partial x} \left( x \cos(\atan x)\right) \frac{\mu(x) \omega}{c}\st_\omega + \left. x \cos(\atan x)\frac{\omega}{c}\frac{\partial \mu(x)}{\partial x}\st_\omega.
\end{equation}
Note that 
\begin{align}
\cos(\atan x) = \frac{1}{\sqrt{(x^2 + 1)}}, & \sin(\atan x) = \frac{x}{\sqrt{(x^2 + 1)}}
\end{align}
then we get
\begin{equation}
\left.\frac{\partial k_\perp}{\partial x}\st_\omega = \left.\frac{1}{(x^2+1)^\frac{3}{2}} \frac{\mu(x) \omega}{c}\st_\omega + \left. \frac{x}{\sqrt{(x^2+1)}} \frac{\omega}{c}\frac{\partial \mu(x)}{\partial x}\st_\omega,
\end{equation}
which becomes
\begin{equation}
\left.\frac{\partial k_\perp}{\partial x}\st_\omega = \left. \frac{\omega}{c \sqrt{(x^2+1)}}\left( \frac{\mu(x)}{(1+x^2)}  + x \frac{\partial \mu(x)}{\partial x}\right)\st_\omega.
\end{equation}

The other term we need is 
\begin{equation} \left.\frac{\partial k_\parallel}{\partial \omega}\right|_{k_\perp}, 
\end{equation}
%where
%\begin{equation}
%k = \sqrt{(k_\parallel^2 + k_\perp^2)} = c \omega \sqrt{ 1 - \frac{\omega_{pe}^2}{\omega(\omega + \cos \theta \Omega_{ce})}}
%\end{equation}
%or
%\begin{equation}
%\frac{c^2 k^2}{\omega^2} =  1 - \frac {\omega_{pe}^2}{\omega(\omega + \Omega_e \cos\theta)}.
%\end{equation}
We can implicitly differentiate $k^2 = k_\parallel^2 + k_\perp^2$ as $ 2 k dk = 2 k_\perp dk_\perp + 2 k_\parallel dk_\parallel$ which for $k_\perp$ held constant means
\begin{equation}
\frac{\partial k}{\partial \omega} = \frac{\partial k_\parallel}{\partial \omega} \frac{k_\parallel}{k}.
\end{equation}Now
%\begin{align}
%\frac{c^2 k^2}{\omega^2} &= \mu^2(\omega)\\
%\frac{\partial \mu}{\partial \omega} &= \frac{- 2 c^2}{\omega^3} \frac{\partial (k^2)}{\partial \omega}\\
%\frac{\partial (k^2)}{\partial \omega}&= 2 k \frac{\partial k}{\partial \omega}.
%\end{align}
\begin{align}
\frac{c^2 k^2}{\omega^2} &= \mu^2(\omega)\\
\frac{\partial \mu}{\partial \omega} &= -\frac{ c k }{\omega^2} + \frac{c}{\omega}\frac{\partial k}{\partial \omega}\\
%\frac{\partial (k^2)}{\partial \omega}&= 2 k \frac{\partial k}{\partial \omega}.
\end{align}
So
\begin{align}
\frac{\partial k_\parallel}{\partial \omega}  &=  \frac{k}{k_\parallel} \frac{\partial k}{\partial \omega}\notag\\
&= \frac{k}{k_\parallel}\left( \frac{\partial \mu}{\partial \omega} + \frac {ck}{\omega^2}\right)\frac{\omega}{c}.
%&=\frac{k}{k_\parallel}  \frac{1}{2k} \frac{\omega^3}{- 2 c^2} \frac{\partial \mu}{\partial \omega}\notag \\
%&= -\frac{1}{ 4k_\parallel}  \frac{\omega^3}{c^2} \frac{\partial \mu}{\partial \omega}
\end{align}

Now we get:
\begin{equation}
N(\omega) = \frac{2  \omega^2}{(2\pi)^2 c^2} \int_0^\infty g_\omega( \tan^{-1} x ) 
 \left. \left( \frac{\mu(x)}{(1+x^2)}  + x \frac{\partial \mu(x)}{\partial x}\right)\st_\omega
\left. \left( \frac{\partial \mu}{\partial \omega} + \frac {ck}{\omega^2}\right) \st_{k_\perp}
 k_\perp d x .
\end{equation}
The final step is to eliminate $k_\perp$ in favour of $\omega, x, \mu(\omega, x)$, using $ k_\perp = \mu \omega \sin(\atan x)/c$ to get 
\begin{equation}
N(\omega) = \frac{2  \omega^3}{(2\pi)^2 c^3} \int_0^\infty g_\omega( \atan x ) 
\frac{x \mu(x, \omega)}{\sqrt{(x^2 + 1)}} \left. \left( \frac{\mu(x)}{(1+x^2)}  + x \frac{\partial \mu(x)}{\partial x}\right)\st_\omega
\left. \left( \frac{\partial \mu}{\partial \omega} + \frac {\mu}{\omega}\right) \st_{\mu \omega \cos(\atan x)}
 d x .
\end{equation}

\section{Obtaining Lyons A7}
Substitute Lyons Eq (12) for Whistler mode (rewritten in terms of $x = \tan \theta$):
\begin{align}
&\mu^2 = \frac{\omega_{pe}^2}{\Omega_c^2}\frac{1+M}{M} \Psi^{-1} =: A \Psi^{-1}\\
&\Psi = 1 - \frac{\omega}{\Omega_p\Omega_e} - \frac{x^2}{2(x^2 + 1)} + \left[\frac{x^4}{4(x^2+1)^2} + \left(\frac{\omega}{\Omega_p}\right)^2(1-M)^2\frac{1}{x^2 +1}\right]^{1/2}
\end{align}
to get \begin{equation}
N(\omega) = \frac{2  \omega^3}{(2\pi)^2 c^3} \int g_\omega( \atan x ) 
\frac{x A^{1/2} \Psi^{-1/2}}{\sqrt{(x^2 + 1)}}  \left( \frac{A^{1/2} \Psi^{-1/2}}{(1+x^2)}  + x A^{1/2}\frac{\partial \Psi^{-1/2}}{\partial x}\right)
 \left( A^{1/2}\frac{\partial  \Psi^{-1/2}}{\partial \omega} + \frac {A^{1/2}\Psi^{-1/2}}{\omega}\right)
 d x 
\end{equation}
\begin{equation}
N(\omega) = \frac{2  \omega^3 A^{3/2}}{(2\pi)^2 c^3} \int g_\omega( \atan x ) 
\frac{x \Psi^{-1/2}}{\sqrt{(x^2 + 1)}}  \left( \frac{\Psi^{-1/2}}{(1+x^2)}  + x \frac{\partial \Psi^{-1/2}}{\partial x}\right)
 \left(\frac{\partial  \Psi^{-1/2}}{\partial \omega} + \frac {\Psi^{-1/2}}{\omega}\right)
 d x 
\end{equation}
Now
\begin{align}
\frac{\partial \Psi^{-1/2}}{\partial X} &= -\frac{1}{2}\Psi^{-3/2} \frac{\partial \Psi}{\partial X}, \:X=\{x, \omega\}
\end{align}
and
\begin{align}
& \frac{\partial \Psi}{\partial x} = - \frac{2 x}{2(x^2+1)} + \frac{2x^3}{2(x^2+1)^2} + \notag\\&\frac{1}{2}\left[\frac{x^4}{4(x^2+1)^2} +\left(\frac{\omega}{\Omega_p}\right)^2(1-M)^2\frac{1}{x^2 +1}\right]^{-1/2} \left( \frac{4 x^3}{4(x^2+1)^2} + \frac{-4 x^5 }{4(x^2+1)^3} + \left(\frac{\omega}{\Omega_p}\right)^2(1-M)^2\frac{-2x}{(x^2 +1)^2}\right)\\
&= - \frac{x}{(x^2+1)} + \frac{x^3}{(x^2+1)^2} + \notag\\&\frac{1}{2}\left[\frac{x^4}{4(x^2+1)^2} +\left(\frac{\omega}{\Omega_p}\right)^2(1-M)^2\frac{1}{x^2 +1}\right]^{-1/2} \left( \frac{x^3}{(x^2+1)^2} - \frac{x^5 }{(x^2+1)^3}  -\left(\frac{\omega}{\Omega_p}\right)^2(1-M)^2\frac{2x}{(x^2 +1)^2}\right)
\end{align}











\end{document}
