\documentclass[]{article}
\usepackage[lmargin = 3cm, rmargin = 3cm]{geometry}
\usepackage{amsmath}
\usepackage{hyperref}
\title{Derivations of Diffusion coefficient factors}

\begin{document}
\tableofcontents

\section{Copies of some equations and their major sources}
Stix 2.45 (sign as for Whistler mode):
\begin{equation}
\mu^2 =  1 - \frac {\omega_{pe}^2}{\omega(\omega + \Omega_e \cos\theta)}
\end{equation}
Doppler resonance condition:
\begin{equation}
\omega -k_\parallel v_\parallel = \frac{n \Omega_e}{\gamma}
\end{equation}



\section{Resonant frequencies in high-density approximation}
Start from Stix Equation 2.45 for the Right hand mode, noting that $\Omega_c$ contains the \emph{sign} of the particle charge:
\begin{equation}\label{A}
\mu^2 =  1 - \frac {\omega_{pe}^2}{\omega(\omega + \Omega_e \cos\theta)}
\end{equation}
along with the resonance condition
\begin{equation}
\omega -k_\parallel v_\parallel = \frac{n \Omega_e}{\gamma}
\end{equation}
The latter is rewritten as
\begin{equation}
k_\parallel = k \cos\theta = \frac{1}{v_\parallel} \left(\omega - \frac{n \Omega_e}{\gamma} \right) \end{equation}
and so as $ \mu = c k /\omega = (c k_\parallel)/( \omega \cos\theta)$ we get
\begin{equation}\label{B}
\mu = \frac{c ( \gamma \omega - n \Omega_e)}{\gamma\omega v_\parallel \cos\theta}.
\end{equation}
Now we equate \ref{A} and \ref{B}:
\begin{equation}
\frac{\omega(\omega + \Omega_e \cos\theta) - \omega_{pe}^2}{\omega(\omega + \Omega_e \cos\theta)} = \frac{c^2}{\gamma^2 \omega^2 v_\parallel^2 \cos^2\theta} (\gamma \omega - n \Omega_e)^2 \end{equation}
which rearranges to
\begin{equation}
(\gamma^2 \omega^2 v_\parallel^2 \cos^2\theta) (\omega(\omega + \Omega_e \cos\theta) - \omega_{pe}^2) = c^2 \omega(\omega + \Omega_e \cos\theta)(\gamma \omega - n \Omega_e)^2. \end{equation}
Cancel one $\omega$ from each side:
\begin{equation}
(\gamma^2 \omega v_\parallel^2 \cos^2\theta) (\omega(\omega + \Omega_e \cos\theta) - \omega_{pe}^2) = c^2 (\omega + \Omega_e \cos\theta)(\gamma \omega - n \Omega_e)^2 \end{equation}
and multiply out:
\begin{equation}
(\gamma^2 \omega v_\parallel^2 \cos^2\theta) (\omega(\omega + \Omega_e \cos\theta) - \omega_{pe}^2) = c^2 (\omega + \Omega_e \cos\theta)(\gamma^2 \omega^2 -2 n \Omega_e \gamma \omega + n^2 \Omega_e^2) \end{equation}
and again
\begin{align}
\gamma^2\omega^3 v_\parallel^2 \cos^2 \theta + \gamma^2 \omega^2 v_\parallel^2 \cos^3\theta\Omega_e - \gamma^2\omega v_\parallel^2\cos^2\theta\omega_{pe}^2 &=\notag\\ c^2\gamma^2 \omega^3 + c^2 \Omega_e\cos\theta \gamma^2\omega^2 - 2 c^2 \omega^2 n\Omega_e\gamma - 2 c^2 n \Omega_e^2 \gamma\omega\cos\theta + c^2 \omega n^2\Omega_e^2 + c^2 n^2\Omega_e^3\cos\theta &
\end{align}

Write $v_\parallel / c = \tilde{v}$ and divide through by $c$:
\begin{align}
\gamma^2\omega^3 \tilde{v}^2 \cos^2 \theta + \gamma^2 \omega^2 \tilde{v}^2 \cos^3\theta\Omega_e - \gamma^2\omega \tilde{v}^2\cos^2\theta\omega_{pe}^2 &=\notag\\ \gamma^2 \omega^3 +  \Omega_e\cos\theta \gamma^2\omega^2 - 2  \omega^2 n\Omega_e\gamma - 2  n \Omega_e^2 \gamma\omega\cos\theta +  \omega n^2\Omega_e^2 +  n^2\Omega_e^3\cos\theta&
\end{align}
Now collect up coefficients of each power:
\begin{align}
\left(\gamma^2 \tilde{v}^2 \cos^2 \theta - \gamma^2   \right)&\omega^3  \notag\\
+\left(\gamma^2 \tilde{v}^2 \cos^3\theta\Omega_e  -\Omega_e\cos\theta \gamma^2  + 2 n\Omega_e\gamma  \right)&\omega^2  \notag\\
+\left(- \gamma^2 \tilde{v}^2\cos^2\theta\omega_{pe}^2  +2  n \Omega_e^2 \gamma\cos\theta  - n^2\Omega_e^2   \right)&\omega \notag\\
+\left(  -n^2\Omega_e^3\cos\theta   \right)&
\end{align}
Or as $\omega , \omega_{pe}, |\Omega_e| \gg 1$ we can make the substitution $ \omega_0 = \omega/\omega_{ref}$ to minimise the differences in size between the coefficients, giving:
\begin{align}
\left(\gamma^2 \tilde{v}^2 \cos^2 \theta - \gamma^2   \right)&(\omega_{ref}\omega_0)^3 \notag \\
+\left(\gamma^2 \tilde{v}^2 \cos^3\theta\Omega_e  -\Omega_e\cos\theta \gamma^2  + 2 n\Omega_e\gamma  \right)&(\omega_{ref}\omega_0)^2  \notag\\
+\left(- \gamma^2 \tilde{v}^2\cos^2\theta\omega_{pe}^2  +2  n \Omega_e^2 \gamma\cos\theta  - n^2\Omega_e^2   \right)&\phantom{(}\omega_{ref}\omega_0\phantom{)^2}  \notag\\
+\left(  -n^2\Omega_e^3\cos\theta   \right)&\phantom{(\omega_ref\omega_0)^2} 
\end{align}
or dividing through by $(\omega_{ref})^3$:
\begin{align}
\left(\gamma^2 \tilde{v}^2 \cos^2 \theta - \gamma^2   \right)&(\omega_0)^3 \notag \\
+\left(\gamma^2 \tilde{v}^2 \cos^3\theta(\Omega_e/\omega_{ref})  -(\Omega_e/\omega_{ref})\cos\theta \gamma^2  + 2 n(\Omega_e/\omega_{ref})\gamma  \right)&(\omega_0)^2  \notag\\
+\left(- \gamma^2 \tilde{v}^2\cos^2\theta(\omega_{pe}/\omega_{ref})^2  +2  n (\Omega_e/\omega_{ref})^2 \gamma\cos\theta  - n^2(\Omega_e/\omega_{ref})^2   \right)&\phantom{(}\omega_0\phantom{)^2}  \notag\\
+\left(  -n^2(\Omega_e/\omega_{ref})^3\cos\theta   \right)&\phantom{(\omega_ref\omega_0)^2} 
\end{align}
which are expected to be order 1 for $n \sim 1$ and $\tilde{v}, \cos\theta \sim 1$.

\section{Normalisation of angular distribution, $N(\omega)$}
\label{NormingAngDistrib}
\newcommand{\atan}{\mathrm{atan}}
\newcommand{\st}{\right|}

We need to calculate the factor $N(\omega)$ in Lyons 1974  J Plasma Phys, 12, 417, Equation A3 or A5 for arbitrary $\mu$. I.e. 
\begin{equation}
N(\omega) = \frac{1}{(2\pi)^2} \int_0^\infty g_\omega( \tan^{-1} x ) \left| J \left( \frac{k_\perp, k_\parallel}{\omega, x} \right)\right| k_\perp d x
\end{equation}
Using Eq A6:
\begin{equation}
J \left( \frac{k_\perp, k_\parallel}{\omega, x} \right) = - \left.\frac{\partial k_\perp}{\partial x}\right|_\omega \cdot \left.\frac{\partial k_\parallel}{\partial \omega}\right|_{k_\perp}, 
\end{equation}

\noindent Using $k_\perp = x k_\parallel$ and $ \mu = c k /\omega$ we get 
\begin{equation}
\frac{\partial k_\perp}{\partial x} = \left. \frac{\partial}{\partial x} \left( x \cos(\atan x) \frac{\mu(x) \omega}{c}\right)\st_\omega.
\end{equation}
Using the chain rule:
\begin{equation}
\frac{\partial k_\perp}{\partial x} = \frac{\partial}{\partial x} \left( x \cos(\atan x)\right) \frac{\mu(x) \omega}{c} + x \cos(\atan x)\frac{\omega}{c}\frac{\partial \mu(x)}{\partial x}.
\end{equation}
Note that 
\begin{align}
\cos(\atan x) = \frac{1}{\sqrt{(x^2 + 1)}}, & \sin(\atan x) = \frac{x}{\sqrt{(x^2 + 1)}}
\end{align}
then we get
\begin{equation}
\frac{\partial k_\perp}{\partial x} = \frac{1}{(x^2+1)^\frac{3}{2}} \frac{\mu(x) \omega}{c} +\frac{x}{\sqrt{(x^2+1)}} \frac{\omega}{c}\frac{\partial \mu(x)}{\partial x}. 
\end{equation}
However, we need to evaluate this at a fixed $\omega$ which means that $ck/\mu = \mathrm{Const}$ and as $k$ is a free variable the $\partial\mu/\partial x$ must be 0. This then becomes
\begin{equation}
\left.\frac{\partial k_\perp}{\partial x}\st_\omega = \frac{\omega}{c \sqrt{(x^2+1)}}\frac{\mu(x)}{(1+x^2)} .
\end{equation}

\noindent The other term we need is 
\begin{equation}\label{kpar_omega} \left.\frac{\partial k_\parallel}{\partial \omega}\right|_{k_\perp}, 
\end{equation}
We can implicitly differentiate $k^2 = k_\parallel^2 + k_\perp^2$ as $ 2 k dk = 2 k_\perp dk_\perp + 2 k_\parallel dk_\parallel$ which for $k_\perp$ held constant means
\begin{equation}
\frac{\partial k}{\partial \omega} = \frac{\partial k_\parallel}{\partial \omega} \frac{k_\parallel}{k}.
\end{equation}Now
\begin{align}
\frac{c k}{\omega} &= \mu(\omega)\\
\frac{\partial \mu}{\partial \omega} &= -\frac{ c k }{\omega^2} + \frac{c}{\omega}\frac{\partial k}{\partial \omega}
\end{align}
So
\begin{align}
\frac{\partial k_\parallel}{\partial \omega}  &=  \frac{k}{k_\parallel} \frac{\partial k}{\partial \omega}\notag\\
&= \frac{k}{k_\parallel}\left( \frac{\partial \mu}{\partial \omega} + \frac {ck}{\omega^2}\right)\frac{\omega}{c}.
\end{align}
Alternately we write $k_\parallel = (k^2 - k_\perp^2)^{1/2}$ and differentiate this by $\omega$ to get the same result. We note that this term is to be evaluated at constant $k_\perp$, or equivalently constant $ x \mu / \sqrt{x^2 +1}$, and that $k/k_\parallel = \sqrt{x^2 +1}$

\noindent Using these we get:
\begin{equation}
N(\omega) = \frac{2  \omega^2}{(2\pi)^2 c^2} \int_0^\infty g_\omega( \tan^{-1} x )
\left| \frac{\mu(x)}{(1+x^2)} \right|
\left| \frac{\partial \mu}{\partial \omega} + \frac {ck}{\omega^2}\right|
 k_\perp d x .
\end{equation}
The final step is to eliminate $k_\perp$ in favour of $\omega, x, \mu(\omega, x)$, using $ k_\perp = \mu \omega \sin(\atan x)/c$ to get 
\begin{equation}\label{N_om}
N(\omega) = \frac{2  \omega^3}{(2\pi)^2 c^3} \int_0^\infty g_\omega( \atan x ) 
\frac{x \mu^2(x, \omega)}{(x^2 + 1)^{3/2}}
 \left| \frac{\partial \mu}{\partial \omega} + \frac {\mu}{\omega}\right|
 d x .
\end{equation}

\section{Obtaining Lyons A7}

Substitute Lyons Eq (12) for Whistler mode (rewritten in terms of $x = \tan \theta$):
\begin{align}
&\mu^2 = \frac{\omega_{pe}^2}{\Omega_c^2}\frac{1+M}{M} \Psi^{-1} =: A \Psi^{-1}\\
&\Psi = 1 - \frac{\omega^2}{\Omega_p\Omega_e} - \frac{x^2}{2(x^2 + 1)} + \left[\frac{x^4}{4(x^2+1)^2} + \left(\frac{\omega}{\Omega_p}\right)^2(1-M)^2\frac{1}{x^2 +1}\right]^{1/2}
\end{align} into \ref{N_om} with $M = m_e /m_p$
to get
\begin{equation}
N(\omega) = \frac{\omega^3}{(2\pi)^2 c^3} \int_{-\infty}^\infty g_\omega( \atan x ) 
 \frac{x}{(1+x^2)^{3/2}} A \Psi^{-1}
  \left( A^{1/2}\frac{\partial  \Psi^{-1/2}}{\partial \omega} + \frac {A^{1/2}\Psi^{-1/2}}{\omega}\right) d x .
\end{equation} Then
\begin{align}
\frac{\partial \Psi^{-1/2}}{\partial \omega} &= -\frac{1}{2}\Psi^{-3/2} \frac{\partial \Psi}{\partial \omega}\\
 \frac{\partial \Psi}{\partial \omega} &= -\frac{2 \omega}{\Omega_p\Omega_e} + \left[\frac{x^4}{4(x^2+1)^2} + \left(\frac{\omega}{\Omega_p}\right)^2(1-M)^2\frac{1}{x^2 +1}\right]^{-1/2}\frac{\omega}{\Omega_p^2}(1-M)^2 \frac{1}{x^2+1}
 \end{align}
 The part in square brackets can be re-written as 
\begin{align}
 \left[\frac{x^4}{4(x^2+1)^2} + \left(\frac{\omega}{\Omega_p}\right)^2(1-M)^2\frac{1}{x^2 +1}\right]^{-1/2} &= \left( \Psi - 1 + \frac{\omega^2}{\Omega_p\Omega_e} + \frac{x^2}{2(x^2+1)}\right)^{-1}\notag\\
 &=(1+x^2) \left[(1+x^2)\left(\Psi - 1 + \frac{\omega^2}{\Omega_p\Omega_e}\right) + \frac{x^2}{2}\right]^{-1}
\end{align}
So
\begin{align}
\left(\frac{\partial  \Psi^{-1/2}}{\partial \omega} + \frac {\Psi^{-1/2}}{\omega}\right) &= \frac{\Psi^{-1/2}}{\omega} \left( 1 +\Psi^{-1} \left[  \frac{ \omega^2}{\Omega_p\Omega_e} - \frac{1}{2}\left[(1+x^2)\left(\Psi - 1 + \frac{\omega^2}{\Omega_p\Omega_e}\right) + \frac{x^2}{2}\right]^{-1} \frac{\omega^2}{\Omega_p^2}(1-M)^2  \right] \right)
\end{align}
And finally
\begin{align}
N(\omega) &= A^{3/2}\frac{\omega^2}{(2\pi)^2 c^3} \int_{-\infty}^\infty g_\omega( x ) 
 \frac{x}{(1+x^2)^{3/2}} ( \Psi)^{-\frac{3}{2}}\times\notag\\
& \left( 1 +\Psi^{-1} \left[  \frac{ \omega^2}{\Omega_p\Omega_e} - \frac{1}{2}\left[(1+x^2)\left(\Psi - 1 + \frac{\omega^2}{\Omega_p\Omega_e}\right) + \frac{x^2}{2}\right]^{-1} \frac{\omega^2}{\Omega_p^2}(1-M)^2  \right] \right)
d x .
\end{align} 

Note that the second factor is defined as $I(\omega)$, 
\begin{align}
I(\omega) := \left( 1 +\Psi^{-1} \left[  \frac{ \omega^2}{\Omega_p\Omega_e} - \frac{1}{2}\left[(1+x^2)\left(\Psi - 1 + \frac{\omega^2}{\Omega_p\Omega_e}\right) + \frac{x^2}{2}\right]^{-1} \frac{\omega^2}{\Omega_p^2}(1-M)^2  \right] \right)
\end{align}

%\begin{align}
%N(\omega) &= \frac{2  \omega^3 A^{3/2}}{(2\pi)^2 c^3} \int g_\omega( \atan x ) 
%\frac{x \Psi^{-1/2}}{\sqrt{(x^2 + 1)}}  \left( \frac{\Psi^{-1/2}}{(1+x^2)}  + x \frac{\partial \Psi^{-1/2}}{\partial x}\right)
% \left(\frac{\partial  \Psi^{-1/2}}{\partial \omega} + \frac {\Psi^{-1/2}}{\omega}\right)
% d x\\ &:= \frac{2  \omega^3 A^{3/2}}{(2\pi)^2 c^3} \int g_\omega( \atan x ) K(x,\omega) dx
%\end{align}
%Now
%\begin{align}
%\frac{\partial \Psi^{-1/2}}{\partial X} &= -\frac{1}{2}\Psi^{-3/2} \frac{\partial \Psi}{\partial X}, \:X=\{x, \omega\}
%\end{align}
%and
%\begin{align}
%& \frac{\partial \Psi}{\partial x} = - \frac{2 x}{2(x^2+1)} + \frac{2x^3}{2(x^2+1)^2} + \notag\\&\frac{1}{2}\left[\frac{x^4}{4(x^2+1)^2} +\left(\frac{\omega}{\Omega_p}\right)^2(1-M)^2\frac{1}{x^2 +1}\right]^{-1/2} \left( \frac{4 x^3}{4(x^2+1)^2} + \frac{-4 x^5 }{4(x^2+1)^3} + \left(\frac{\omega}{\Omega_p}\right)^2(1-M)^2\frac{-2x}{(x^2 +1)^2}\right)\\
%&= - \frac{x}{(x^2+1)} + \frac{x^3}{(x^2+1)^2} + \notag\\&\frac{1}{2}\left[\frac{x^4}{4(x^2+1)^2} +\left(\frac{\omega}{\Omega_p}\right)^2(1-M)^2\frac{1}{x^2 +1}\right]^{-1/2} \left( \frac{x^3}{(x^2+1)^2} - \frac{x^5 }{(x^2+1)^3}  -\left(\frac{\omega}{\Omega_p}\right)^2(1-M)^2\frac{2x}{(x^2 +1)^2}\right)\\
%&= \frac{x}{(x^2+1)^2} \left\{- 1 + \frac{1}{2}\left[\frac{x^4}{4(x^2+1)^2} +\left(\frac{\omega}{\Omega_p}\right)^2(1-M)^2\frac{1}{x^2 +1}\right]^{-1/2} \left( \frac{x^{2?}}{(x^2+1)} -2\left(\frac{\omega}{\Omega_p}\right)^2(1-M)^2\right)\right\}.
%\end{align}
%Also
%\begin{align}
% &\frac{\partial \Psi}{\partial \omega} = -\frac{2 \omega}{\Omega_p\Omega_e} + \left[\frac{x^4}{4(x^2+1)^2} + \left(\frac{\omega}{\Omega_p}\right)^2(1-M)^2\frac{1}{x^2 +1}\right]^{-1/2}\frac{\omega}{\Omega_p^2}(1-M)^2 \frac{1}{x^2+1}.
%\end{align}
%Now we plop these into $K(x, \omega)$ and find
%\begin{align}
%K(x, \omega)&=\frac{x \Psi^{-1/2}}{\sqrt{(x^2 + 1)}}  \left( \frac{\Psi^{-1/2}}{(1+x^2)}  + x \frac{\partial \Psi^{-1/2}}{\partial x}\right)
% \left(\frac{\partial  \Psi^{-1/2}}{\partial \omega} + \frac {\Psi^{-1/2}}{\omega}\right)\\
% &=\Psi^{-1}\frac{1}{\omega} \frac{x}{(x^2+1)^{3/2}}\times\notag\\& \left(1-\frac{1}{2} \Psi^{-1}x \frac{x}{(x^2+1)} \left\{- 1 + \frac{1}{2}\left[\frac{x^4}{4(x^2+1)^2} +\left(\frac{\omega}{\Omega_p}\right)^2(1-M)^2\frac{1}{x^2 +1}\right]^{-1/2} \left( \frac{x^2}{(x^2+1)} -2\left(\frac{\omega}{\Omega_p}\right)^2(1-M)^2\right)\right\}\right)\times\notag\\
% & \left( 1 + \frac{\omega}{2}\Psi^{-1} \left\{\frac{2\omega}{\Omega_p\Omega_e} + \left[\frac{x^4}{4(x^2+1)^2} + \left(\frac{\omega}{\Omega_p}\right)^2(1-M)^2\frac{1}{x^2 +1}\right]^{-1/2}\frac{\omega}{\Omega_p^2}(1-M)^2 \frac{1}{x^2+1}\right\}  \right)
%\end{align}
%Define $K'$ as without the prefactor $ \Psi^{-1}\frac{1}{\omega} \frac{x}{(x^2+1)^{3/2}}$
%\begin{align}
%K'(x, \omega)&= \left(1 - \frac{1}{2} \Psi^{-1} \frac{x^2}{(x^2+1)} \left\{- 1 + \frac{1}{2}\left[\frac{x^4}{4(x^2+1)^2} +\left(\frac{\omega}{\Omega_p}\right)^2(1-M)^2\frac{1}{x^2 +1}\right]^{-1/2} \left( \frac{x^2}{(x^2+1)} -2\left(\frac{\omega}{\Omega_p}\right)^2(1-M)^2\right)\right\}\right)\times\notag\\
% & \left( 1 + \frac{\omega}{2}\Psi^{-1} \left\{\frac{2\omega}{\Omega_p\Omega_e} + \left[\frac{x^4}{4(x^2+1)^2} + \left(\frac{\omega}{\Omega_p}\right)^2(1-M)^2\frac{1}{x^2 +1}\right]^{-1/2}\frac{\omega}{\Omega_p^2}(1-M)^2 \frac{1}{x^2+1}\right\}  \right)
%\end{align}
%The part in square brackets can be re-written as 
%\begin{align}
% \left[\frac{x^4}{4(x^2+1)^2} + \left(\frac{\omega}{\Omega_p}\right)^2(1-M)^2\frac{1}{x^2 +1}\right]^{-1/2} &= \left( \Psi - 1 + \frac{\omega^2}{\Omega_p\Omega_e} + \frac{x^2}{2(x^2+1)}\right)^{-1}\notag\\
% &=(1+x^2) \left[(1+x^2)\left(\Psi - 1 + \frac{\omega^2}{\Omega_p\Omega_e}\right) + \frac{x^2}{2}\right]^{-1}
%\end{align}
%so multiplying out we get
%\begin{align}
%K'(x, \omega)&= 1 + \Psi^{-1} \frac{\omega^2}{\Omega_p\Omega_e} + \Psi^{-1}\frac{x^2}{2(x^2+1)}\notag\\ 
%%& -\frac{1}{2} (1+x^2) \left[(1+x^2)\left(\Psi - 1 + \frac{\omega^2}{\Omega_p\Omega_e}\right) + \frac{x^2}{2}\right]^{-1}\frac{\omega^2}{\Omega_p^2}(1-M)^2 \frac{1}{1+x^2} \notag\\&+ \frac{1}{2} (1+x^2) \left[(1+x^2)\left(\Psi - 1 + \frac{\omega^2}{\Omega_p\Omega_e}\right) + \frac{x^2}{2}\right]^{-1}\left( \frac{x}{(x^2+1)} -2\left(\frac{\omega}{\Omega_p}\right)^2(1-M)^2\right)\notag\\
%& +\Psi^{-1}\frac{1}{2} \left[(1+x^2)\left(\Psi - 1 + \frac{\omega^2}{\Omega_p\Omega_e}\right) + \frac{x^2}{2}\right]^{-1}\frac{\omega^2}{\Omega_p^2}(1-M)^2 \notag\\&+ \Psi^{-1}\frac{1}{4}	x^2 \left[(1+x^2)\left(\Psi - 1 + \frac{\omega^2}{\Omega_p\Omega_e}\right) + \frac{x^2}{2}\right]^{-1}\left( \frac{x^2}{(x^2+1)} -2\left(\frac{\omega}{\Omega_p}\right)^2(1-M)^2\right)\notag\\
%& - \frac{1}{4}\Psi^{-2} \frac{x^2}{2(x^2+1)} \left\{-1 +\frac{1}{2}\left[\frac{x^4}{4(x^2+1)^2} +\left(\frac{\omega}{\Omega_p}\right)^2(1-M)^2\frac{1}{x^2 +1}\right]^{-1/2} \left( \frac{x^2}{(x^2+1)} -2\left(\frac{\omega}{\Omega_p}\right)^2(1-M)^2\right)\right\}\times\notag\\
%&\left\{\frac{2\omega^2}{\Omega_p\Omega_e} + \left[\frac{x^4}{4(x^2+1)^2} + \left(\frac{\omega}{\Omega_p}\right)^2(1-M)^2\frac{1}{x^2 +1}\right]^{-1/2}\frac{\omega^2}{\Omega_p^2}(1-M)^2 \frac{1}{x^2+1}\right\}.
%%& + \left[(1+x^2)\left(\Psi - 1 + \frac{\omega^2}{\Omega_p\Omega_e}\right) + \frac{x^2}{2}\right]^{-1}\left(\frac{\omega}{\Omega_p^2}(1-M)^2 \frac{x}{2} -(1+x^2)\left(\frac{\omega}{\Omega_p}\right)^2(1-M)^2\right)\notag\\
%\end{align}
%We also have
%\begin{align}
%fff 
%\end{align}
%
%
%% \begin{align}
%% K'(x, \omega)&= 1 + \Psi^{-1} \frac{\omega^2}{2\Omega_p\Omega_e} + \Psi^{-1}\frac{x^2}{2(x^2+1)}\notag\\ 
%% & +\Psi^{-1}\frac{1}{2} \left[(1+x^2)\left(\Psi - 1 + \frac{\omega^2}{\Omega_p\Omega_e}\right) + \frac{x^2}{2}\right]^{-1}\frac{\omega^2}{\Omega_p^2}(1-M)^2(1-x^2) \notag\\&+ \Psi^{-1}\frac{1}{4}	x^2 \left[(1+x^2)\left(\Psi - 1 + \frac{\omega^2}{\Omega_p\Omega_e}\right) + \frac{x^2}{2}\right]^{-1} \frac{x^2}{(x^2+1)}\notag\\
%% & - \frac{1}{4}\Psi^{-2} \frac{x^2}{2(x^2+1)} \left\{-1 +\frac{1}{2}\left[\frac{x^4}{4(x^2+1)^2} +\left(\frac{\omega}{\Omega_p}\right)^2(1-M)^2\frac{1}{x^2 +1}\right]^{-1/2} \left( \frac{x^2}{(x^2+1)} -2\left(\frac{\omega}{\Omega_p}\right)^2(1-M)^2\right)\right\}\times\notag\\
%% &\left\{\frac{2\omega^2}{\Omega_p\Omega_e} + \left[\frac{x^4}{4(x^2+1)^2} + \left(\frac{\omega}{\Omega_p}\right)^2(1-M)^2\frac{1}{x^2 +1}\right]^{-1/2}\frac{\omega^2}{\Omega_p^2}(1-M)^2 \frac{1}{x^2+1}\right\}
%% \end{align}
%% 
%
%%\begin{align}
%%$K'(x, \omega)&= 1 - \Psi^{-1} \frac{\omega^2}{\Omega_p\Omega_e} - \Psi^{-1}\frac{x^2}{x^2+1}\notag\\ 
%%& -\Psi^{-1}\frac{3}{2} \left[(1+x^2)\left(\Psi - 1 + \frac{\omega^2}{\Omega_p\Omega_e}\right) + \frac{x^2}{2}\right]^{-1}\frac{\omega^2}{\Omega_p^2}(1-M)^2 \notag\\&+\Psi^{-1} \frac{1}{2} \left[(1+x^2)\left(\Psi - 1 + \frac{\omega^2}{\Omega_p\Omega_e}\right) + \frac{x^2}{2}\right]^{-1}\left(x  -2 x^2\left(\frac{\omega}{\Omega_p}\right)^2(1-M)^2\right)\notag\\
%%& +\Psi^{-2}...
%%\end{align}
%%Now we consider the terms
%%\begin{align}
%%\frac{x^2}{x^2+1}+  \frac{1}{2} \left[(1+x^2)\left(\Psi - 1 + \frac{\omega^2}{\Omega_p\Omega_e}\right) + \frac{x^2}{2}\right]^{-1}\left(x  -2 x^2\left(\frac{\omega}{\Omega_p}\right)^2(1-M)^2\right)\notag\\
%%=\frac{1}{x^2+1} \left(x^2 + (1+x^2)^2 \left(x  -2 x^2\left(\frac{\omega}{\Omega_p}\right)^2(1-M)^2\right)\right) \left[\frac{x^4}{4(x^2+1)^2} + \left(\frac{\omega}{\Omega_p}\right)^2(1-M)^2\frac{1}{x^2 +1}\right]^{-1/2} 
%%\end{align}
%

\section{G2 transformation to tan axis}
G2 in Albert (2005) is integral over theta, but in our case we have a $\tan(\theta)$ axis so we transform 
\begin{equation}
x = \tan(\theta)
\end{equation}
so
\begin{equation}
\frac{\mathrm{d} \tan(\theta)}{\mathrm{d} \theta} = \sec^2(\theta) = \frac{1}{\cos^2(\theta)}
\end{equation}
i.e.
\begin{equation}
\mathrm{d} \theta = \cos^2(\theta) \mathrm{d}\tan\theta
\end{equation}
providing $\cos\theta \ne 0$. Similarly, the $\sin\theta$ term is 
\begin{equation}
\sin\theta = \tan \theta \cos\theta 
%= \frac{\tan\theta}{\sqrt{\tan^2 \theta + 1}}
%\tan^2\theta + 1 = \sec^2(\theta) = \frac{1}{\cos^2(\theta)}
\end{equation}
so using $ \tan^2\theta + 1 = \sec^2\theta$ we get
\begin{equation}\label{TanTransform}
\sin\theta \mathrm{d}\theta = \tan\theta \cos^3\theta \mathrm{d}\tan\theta= \frac{\tan\theta}{(\tan^2 \theta + 1)^{3/2}}\mathrm{d}\tan\theta = \frac{x}{(x^2+1)^{3/2}} \mathrm{d}x
\end{equation}
Then Albert's Eq 3 for G2 becomes
\begin{equation}
G_2 = g_\omega(\theta)/\left[\int g_\omega(\atan(\theta^\prime)) \mu^2 \left| \mu + \omega \frac{\partial \mu}{\partial\omega} \right| \frac{x}{(x^2+1)^{3/2}} \mathrm{d}x\right]
\end{equation}

\section{Relation of Alberts G1 and G2 to Lyons equations}
Consider Albert Eq 2 (Al2) which is
\begin{align}
D_{\alpha\alpha}^{nx} &= \frac{B_{wave}^2}{B_0^2}(mv)^2 \frac{\pi}{2}\frac{\Omega_c^2}{\Delta\omega}\frac{\cos^2\theta}{|v_\parallel/c|^3}  \Phi_n^2 \frac{(-\sin^2\alpha + \omega_n/\omega)^2}{|1-(\partial \omega/\partial k_\parallel)_x/v_\parallel |} G_1 G_2\\
G_1(\omega) &= \frac{(\Delta \omega)B^2(\omega)}{\int B^2(\omega^\prime) \mathrm{d} \omega^\prime}\\
G_2(\omega, \theta) &= \frac{g_\omega(\theta)}{\int g_\omega(\theta^\prime) \mu^2 | \mu + \omega \partial\mu/\partial\omega| \sin \theta^\prime \mathrm{d} \theta^\prime}\label{Alb_G_2}
\end{align}
noting that we have restored the normalisation factors from
\begin{equation}
D_{xx} = \Omega_c (B_{wave}^2/B_0^2)(mv)^2 D^{norm}_{xx}
\end{equation}
and that
\begin{equation}
B_{wave}^2 = \int B^2(\omega^\prime)\mathrm{d}\omega^\prime
\end{equation}
$\Phi_n^2$ is as in Lyons, Eq 9 (Ly9) and contains some Bessel functions etc etc\\
Ly2 should match Al2. Take Ly2 and subs for $\Theta$ as given before 9, and get $|B|$ from the appendix:
\begin{align}
D_{\alpha\alpha}^{nk_\perp} &= \mathrm{lim}_{V\rightarrow \infty} \frac{\pi q^2}{(2\pi)^2 V m^2 v_\parallel} \left[ \frac{-\sin^2\alpha + n \Omega/\omega_k}{\cos \alpha}\right]^2 \frac{ |B_k|^2 |\Phi_{n,k}|^2}{\mu^2 |1-(\partial \omega_k/\partial k_\parallel)_x/v_\parallel |}\\
|B_k|^2 &= \frac{V}{N(\omega)}B^2(\omega)g_\omega(\theta)
\end{align}
where $q, m$ are the (per species) charge and mass respectively. Derivation of $N(\omega)$ following Lyons was done above. 

They look fairly consistent, so let's rearrange the Lyons one to match the Albert:
\begin{align}
D_{\alpha\alpha}^{nk_\perp} &= \mathrm{lim}_{V\rightarrow \infty} \frac{\pi q^2}{(2\pi)^2 V m^2 v_\parallel} \left[ \frac{-\sin^2\alpha + n \Omega/\omega_k}{\cos \alpha}\right]^2 \frac{ |\Phi_{n,k}|^2}{\mu^2 |1-(\partial \omega_k/\partial k_\parallel)_x/v_\parallel|} \frac{V}{N(\omega)}B^2(\omega)g_\omega(\theta)
\end{align}
Cancel the $V$ and subs the prefactors of $N$ from Eq \ref{N_om}, i.e. define
\begin{equation} \label{N_prime} N(\omega) =: N^\prime(\omega) \frac{2\omega^3}{(2\pi)^2 c^3}\end{equation}
\begin{align}
D_{\alpha\alpha}^{nk_\perp} &= \frac{\pi q^2}{2 m^2 v_\parallel} \left[ \frac{-\sin^2\alpha + n \Omega/\omega_k}{\cos \alpha}\right]^2 \frac{ |\Phi_{n,k}|^2}{\mu^2 |1-(\partial \omega_k/\partial k_\parallel)_x/v_\parallel|} \frac{1}{N^\prime(\omega)}B^2(\omega)g_\omega(\theta)\frac{ c^3}{\omega^3}
\end{align}
$\cos^2 \alpha$ can be replaced by $(v_\parallel/v)^2$ to get
\begin{align}\label{LyMid}
D_{\alpha\alpha}^{nk_\perp} &= \frac{\pi q^2 v^2}{2 m^2 (v_\parallel/c)^3 \omega^3} \frac{1}{\mu^2} |\Phi_{n,k}|^2 \frac{ (-\sin^2\alpha + n \Omega/\omega_k)^2}{ |1-(\partial \omega_k/\partial k_\parallel)_x/v_\parallel|} \frac{1}{N^\prime(\omega)}B^2(\omega)g_\omega(\theta)
\end{align}

\noindent Tackle G1 and G2 next. Starting from Eq \ref{N_om} and Eq \ref{N_prime}
\begin{equation}
N^\prime(\omega) =\int_{-\infty}^\infty g_\omega( \atan x ) 
 \frac{\mu(x)^2 x}{(1+x^2)^{3/2}}
 \left| \frac{\partial \mu}{\partial \omega} + \frac {\mu}{\omega}\right|
 d x
\end{equation} and Eq \ref{Alb_G_2} with Eq \ref{TanTransform} which gives a denominator of
\begin{align}
G2_{denom} &= \int g_\omega(\theta^\prime) \mu^2 \left| \mu + \omega \frac{\partial\mu}{\partial\omega}\right| \sin \theta^\prime \mathrm{d} \theta^\prime\\
&=\int g_\omega(\atan(x^\prime)) \mu^2 \left| \mu + \omega \frac{\partial\mu}{\partial\omega}\right|  \frac{x^\prime}{((x^\prime)^2 + 1)^{3/2}}\mathrm{d}x^\prime
\end{align}
i.e.
\begin{align}
G2_{denom} = \omega N^\prime(\omega)
\end{align}
For G1 we just absorb the $B^2_{wave}$ and $\Delta \omega$ and so Eq \ref{LyMid} becomes
\begin{align}
D_{\alpha\alpha}^{nk_\perp} &= \frac{\pi q^2 v^2}{2 m^2 (v_\parallel/c)^3 \omega^3} \frac{1}{\mu^2} |\Phi_{n,k}|^2 \frac{ (-\sin^2\alpha + n \Omega/\omega_k)^2}{ |1-(\partial \omega_k/\partial k_\parallel)_x/v_\parallel|} \frac{G_1}{\Delta \omega} B^2_{wave} \omega G_2
\end{align}
Recalling
\begin{align}
\frac{c k}{\omega} &= \mu(\omega) %= \frac{c k_\parallel}{ \omega \cos\theta}\\
%\omega -k_\parallel v_\parallel &= \frac{n \Omega_e}{\gamma}
\end{align} we get
\begin{align}\label{LyMid2}
D_{\alpha\alpha}^{nk_\perp} &= \frac{\pi q^2 v^2}{2 m^2 (v_\parallel/c)^3}\frac{1}{c^2 k^2} |\Phi_{n,k}|^2 \frac{ (-\sin^2\alpha + n \Omega/\omega_k)^2}{ |1-(\partial \omega_k/\partial k_\parallel)_x/v_\parallel|} \frac{G_1}{\Delta \omega} B^2_{wave} G_2
\end{align}
Then using $ \Omega_c = q B_0 / m_e c$ we find
\begin{align}\label{LyMid3}
D_{\alpha\alpha}^{nk_\perp} &= \frac{B^2_{wave}}{B_0^2} \frac{\pi v^2 \Omega_c^2}{2 \Delta \omega(v_\parallel/c)^3}\frac{1}{ k^2} |\Phi_{n,k}|^2 \frac{ (-\sin^2\alpha + n \Omega/\omega_k)^2}{ |1-(\partial \omega_k/\partial k_\parallel)_x/v_\parallel|} G_1 G_2
\end{align}
However, this is the expression for $D(k_\perp)$ while Albert has $D(x)$, so comparing Ly1 to Alb5 implies
\begin{equation}
D_{\alpha\alpha} = \int x D(x) \mathrm{d}x = \int k_\perp D(k_\perp) \mathrm{d} k_\perp
\end{equation}
and we must multiply the Lyons result by $k_\perp/x dk_\perp/dx$. 
We have $k_\perp = k \sin \theta = k x /\sqrt(x^2 +1)$ so $dk_\perp/dx = k (x^2+1)^{-3/2}$ and
\[ \frac{k_\perp}{x}\frac{\mathrm{d} k_\perp}{\mathrm{d} x} = \frac{k^2}{(x^2+1)}  = k^2 \cos^4\theta\]
so we end up with, after reordering terms
\begin{align}\label{LyEnd}
D_{\alpha\alpha}^{nx} &= \frac{B^2_{wave}}{B_0^2} \frac{\pi}{2} v^2 \frac{ \Omega_c^2 }{\Delta \omega}\frac{\cos^4\theta}{(v_\parallel/c)^3} |\Phi_{n,k}|^2 \frac{ (-\sin^2\alpha + n \Omega/\omega_k)^2}{ |1-(\partial \omega_k/\partial k_\parallel)_x/v_\parallel|} G_1 G_2
\end{align}

NB NB NB we have some erroneous $m^2$ and a $cos^4$ instead of $cos^2$, compared to the Albert result which is
\begin{align}
D_{\alpha\alpha}^{nx} &= \frac{B_{wave}^2}{B_0^2}(mv)^2 \frac{\pi}{2}\frac{\Omega_c^2}{\Delta\omega}\frac{\cos^2\theta}{|v_\parallel/c|^3}  \Phi_n^2 \frac{(-\sin^2\alpha + \omega_n/\omega)^2}{|1-(\partial \omega/\partial k_\parallel)_x/v_\parallel|} G_1 G_2 \end{align}

NB NB I think this is because they use $g(\theta)$ vs $g(\tan(\theta))$ respectively!!

\section{Calculation of D}

Since D is evaluated at each $p, \alpha$ and to do so requires integrating over wave angle and totalling over both $n$ and the resonant omegas we have 4 or 5 nested loops unavoidably. Therefore we break the calculation down and hoist everything as high as possible. Taking
\begin{align}
D_{\alpha\alpha}^{nx} &= x \frac{B_{wave}^2}{B_0^2}(mv)^2 \frac{\pi}{2}\frac{\Omega_c^2}{\Delta\omega}\frac{\cos^2\theta}{|v_\parallel/c|^3}  \Phi_n^2 \frac{(-\sin^2\alpha + \omega_n/\omega)^2}{|1-(\partial \omega/\partial k_\parallel)_x/v_\parallel|} G_1 G_2 \end{align}
we break the terms down as follows. Note that $\omega$ is the resonant solution in the innermost loop

\begin{itemize}
\item Constant factors, \[ \frac{\pi \Omega_c^2 c^3 m^2}{2 B_0^2}\] This $m^2$ I can't match with Lyons though, see above....
\item Particle parallel momentum $v_\parallel$, \[\frac{1}{v_\parallel^3}\]
\item Particle pitch angle $\alpha$, which gives the \[ v^2 = \frac{v^2_\parallel}{\cos^2 \alpha}\]
\item Wave normal angle, which we keep as $x= \tan\theta$ and integrate over and includes $x \cos^2\atan(x)$
\item Resonance number, $n$
\item Resonant omega solution, $\omega$ in the above equation and dictating anything containing $\mu$ also, i.e. 
\[ \Phi_n^2 \frac{(-\sin^2\alpha + \omega_n/\omega)^2}{|1-(\partial \omega/\partial k_\parallel)_x/v_\parallel|} (G_1 B_{wave}^2/\Delta\omega) G_2 \]

\end{itemize}

\section{Really important subtleties that often get glossed over}

\subsection{Arguments of Bessel functions in Phi}
In Lyons (between Eq 3 and 4) we have that the Bessel function args Big-Theta or Phi are
\[ k_\perp v_\perp / \Omega_l\]
Albert converts these to 
\[ n x \tan \alpha (\omega -\omega_n)/\omega_n\] which DOES NOT WORK for $n = 0$. Assuming that the $n/\omega_n$ is introduced in that form, we can simply cancel the $n$. Alternately perhaps we want a different arg when $n=0$ which is a Landau resonance. 

\subsection{Sign of cyclotron frequency}

Lyons considers a signed $\Omega_l$ but an unsigned $\Omega_e$. Albert has only unsigned $\Omega_c$. 

\subsection{Sign of resonant omega and multiple root counting}
The previous subsection and the fact that negative wave frequency $\omega$ is mostly ignored means I don't know which solutions should be considered. In the resonance condition
\[ \omega  - k_\parallel v_\parallel = - \frac{n |\Omega_c|}{\gamma} \] we get plus and minus signs for the same omega value for pairwise sign swaps of $k_\parallel$, $v_\parallel$ and $n$. It's not clear in Albert whether these things should be signed or whether sign is absorbed into angle, but in Lyons with the $\perp \parallel$ breakdown both must be signed. 

\section{Bounce averaging of D}
The equations for bounce averaging the various D components are:

\begin{align}\label{Bounce_time}
\left< D_{\alpha \alpha}\right> &=\frac{1}{\tau_B} \int_0^{\tau_B} D_{\alpha \alpha}(\alpha) \left(\frac{\partial \alpha_{eq}}{\partial \alpha}\right)^2 \mathrm{d}t\\
\left< D_{\alpha p}\right> &=\frac{1}{\tau_B} \int_0^{\tau_B} D_{\alpha p}(\alpha) \left(\frac{\partial \alpha_{eq}}{\partial \alpha}\right) \mathrm{d}t\\
\left< D_{p p}\right> &=\frac{1}{\tau_B} \int_0^{\tau_B} D_{p p}(\alpha) \mathrm{d}t
\end{align}
with $\tau_B$ the bounce period of the particle, $\tau_B = 4 R_0 S(\alpha_{eq})/v_0$ and $S(\alpha_{eq}) \simeq 1.3 - 0.56 \sin\alpha_{eq}$
\\Using a dipole field, we have the field strength and particle pitch angles given by
\begin{align}
B &= B_{eq} f(\lambda)\\
f(\lambda) &= \frac{(1+3\sin^2\lambda)^{1/2}}{\cos^6\lambda}\label{bounce_f}\\
\sin^2\alpha &= f(\lambda) \sin^2\alpha_{eq}\label{bounce_f2}
\end{align}
Then the time integrals can be converted to latitude integrals using
\begin{equation}
v_0 \cos \alpha \mathrm{d} t = R_0 \cos \lambda (1+3\sin^2\lambda)^{1/2} \mathrm{d}\lambda
\end{equation}
although the $\cos \alpha$ factor makes them potentially degenerate at the mirror point. The equations are then
\begin{align}\label{Bounce_lam}
\left< D_{\alpha \alpha}\right> &=\frac{1}{S(\alpha_{eq})} \int_0^{\lambda_m} D_{\alpha \alpha}(\alpha) \frac{\cos \alpha \cos^7\lambda}{\cos^2\alpha_{eq}} \mathrm{d}\lambda\\
\left< D_{\alpha p}\right> &=\frac{1}{S(\alpha_{eq})} \int_0^{\lambda_m} D_{\alpha p}(\alpha) \frac{\sin \alpha \cos^7\lambda}{\sin \alpha_{eq}\cos\alpha_{eq}} \mathrm{d}\lambda\\
\left< D_{p p}\right> &=\frac{1}{S(\alpha_{eq})} \int_0^{\lambda_m} D_{p p}(\alpha) \frac{\sin^2 \alpha \cos^7\lambda}{\sin^2 \alpha_{eq}\cos\alpha} \mathrm{d}\lambda
\end{align}
NB the third equation has a $\cos\alpha$ in the denominator and is thus ill defined at the mirror point $\lambda_m$
Using \ref{bounce_f} and \ref{bounce_f} we can rewrite these as
\begin{align}\label{Bounce_lam_2}
\left< D_{\alpha \alpha}\right> &=\frac{1}{S(\alpha_{eq})} \int_0^{\lambda_m} D_{\alpha \alpha}(\alpha) \frac{\cos \alpha \cos^7\lambda}{\cos^2\alpha_{eq}} \mathrm{d}\lambda\\
\left< D_{\alpha p}\right> &=\frac{1}{S(\alpha_{eq})} \int_0^{\lambda_m} D_{\alpha p}(\alpha) \frac{\cos^4\lambda (1+3\sin^2\lambda)^{1/4}}{\cos\alpha_{eq}} \mathrm{d}\lambda\\
\left< D_{p p}\right> &=\frac{1}{S(\alpha_{eq})} \int_0^{\lambda_m} D_{p p}(\alpha) \frac{\cos\lambda (1+3\sin^2\lambda)^{1/2}}{\cos\alpha} \mathrm{d}\lambda\label{Bounce_lam_2_c}
\end{align}

\subsection{Analytic integrals for bounce-average testing}
We can substitute a suitable form of $D$ to allow the analytic evaluation of the bounce averaging. While we can always consume the entire $\alpha$ dependence that way, it's more useful to evaluate slightly different things in each case.
For the $\alpha \alpha$ coefficient we use the original Eq \ref{Bounce_lam}:
\begin{align}
D_{\alpha\alpha} &= \frac{1}{\cos \alpha}\\
\left< D_{\alpha \alpha}\right> &= \frac{1}{S(\alpha_{eq})} \frac{1}{\cos^2 \alpha_{eq}} \left[ 1225 \sin \lambda + 245 \sin 3\lambda + 49 \sin 5\lambda + 5 \sin 7\lambda\right]/2240
\end{align}
similarly the mixed coefficient:
\begin{align}
D_{\alpha p} &= \frac{1}{\sin \alpha}\\
\left< D_{\alpha p}\right> &= \frac{1}{S(\alpha_{eq})} \frac{1}{\cos\alpha_{eq}\sin\alpha_{eq}} \left[ 1225 \sin \lambda + 245 \sin 3\lambda + 49 \sin 5\lambda + 5 \sin 7\lambda\right]/2240
\end{align}
For the $p p$ we use the rewritten component Eq \ref{Bounce_lam_2_c}:
\begin{align}
D_{p p} & = \cos \alpha\\
\left< D_{p p}\right> &= \frac{1}{S(\alpha_{eq})} \frac{1}{6} \left[ 3 \sin \lambda \sqrt{3 \sin^2 \lambda + 1} + \sqrt{3} \mathrm{sinh}^{-1}(\sqrt{3}\sin\lambda)\right]
\end{align}

\newpage
\appendix
\section{Failing attempt to derive Lyons N factor using IBP}
Since we have a term $k_\perp  \frac{\partial k_\perp}{\partial x}= \frac{\partial (k_\perp^2)}{2 \partial x}$ it suggests we can use integration by parts on this. We need
\begin{align}
N(\omega) = \frac{2}{(2\pi)^2} \int_0^\infty g_\omega( \tan^{-1} x ) \left| \frac{\partial k_\perp}{\partial x}k_\perp\right|_\omega \left| \frac{\partial k_\parallel}{\partial \omega}\right|_{k_\perp}   d x
\end{align}
%Using $k_\perp = x k_\parallel$ and $ \mu = c k /\omega$ we get
%\begin{align}
%k_\perp \frac{\partial k_\perp}{\partial x} &= \frac{\partial (k_\perp^2)}{2 \partial x}
%\end{align}

The other term is as in \ref{NormingAngDistrib}.

Using the standard expression for 3-term IBP, i.e.
\begin{align}\label{IBP}
\int_a^b u v dw = [uvw]_a^b - \int_a^b uw dv - \int_a^b vw du
\end{align} we put
\begin{align}
u&= \sqrt{x^2 +1} \left.\left( \frac{\partial \mu}{\partial \omega} + \frac {\mu}{\omega}\right) \st_{k_\perp}\\
v&=g(x)\\
dw &= \frac{\partial}{\partial x}\left(\frac{x^2 \mu^2}{x^2 +1}\right)
\end{align}
Then we need $\frac{\partial u}{\partial x}$. But we know that in $u$ $ x \mu / \sqrt{x^2 +1}$ is constant. So
\begin{align}
\frac{\partial u}{\partial x} &= \frac{\partial}{\partial x} \left[ \frac{x^2+1}{x} \left( \frac{\partial}{\partial \omega} \frac{x \mu}{\sqrt{x^2+1}}  + \frac{x \mu}{\omega\sqrt{x^2+1}} \right)\right]\\
&=\frac{\partial}{\partial x} \left(\frac{x^2+1}{x} \right) \left( \frac{\partial}{\partial \omega} \frac{x \mu}{\sqrt{x^2+1}}  + \frac{x \mu}{\omega\sqrt{x^2+1}} \right)
\end{align}
And $w$ is trivially integrated.
%\begin{align}
%\int_0^\infty \left| \frac{\partial k_\perp}{\partial x}k_\perp\right|_\omega \left| \frac{\partial k_\parallel}{\partial \omega}\right|_{k_\perp}   d x &= \int_0^\infty \frac{x^2 \mu^2}{x^2 +1} \frac{x^2-1}{x^2}  \frac{x}{\sqrt{x^2 +1}} \left( \frac{\partial \mu}{\partial \omega} + \frac{ \mu}{\omega} \right)  d x\\
%&=\int_0^\infty \frac{\mu^2 (x^2-1) x}{(x^2 +1)^{3/2}} \left( \frac{\partial \mu}{\partial \omega} + \frac{ \mu}{\omega} \right)  d x
%\end{align}
Now taking terms one by one in \ref{IBP} and assuming well behaved $\mu, g(x)$ and $g(x)\rightarrow 0, x\rightarrow \infty$ and symmetry of $g$, hence antisymmetry of $dg/dx$:
\begin{align}
[uvw] &= 0 \\
\int u w dv &= \int f(x^2) \frac{dg}{dx}\notag\\
&= 0 \; \left(f(x^2) \;\mathrm{symmetric \,by\, definition}\: \frac{dg}{dx} \;\mathrm{antisymmetric}\right)\\
\int vw du &= \int g(x) \frac{\mu^2 (x^2-1) x}{(x^2 +1)^{3/2}} \left( \frac{\partial \mu}{\partial \omega} + \frac{ \mu}{\omega} \right)  d x
\end{align}
Then the full expression becomes \begin{equation}\label{N_IBP}
N(\omega) = \frac{\omega^3}{(2\pi)^2 c^3} \int_{-\infty}^\infty g_\omega( \atan x ) 
 \frac{\mu(x)^2 x (x^2-1)}{(1+x^2)^{3/2}}
 \left| \frac{\partial \mu}{\partial \omega} + \frac {\mu}{\omega}\right|
 d x .
\end{equation}

*******
NB NB NB NB We expected instead this:
\begin{equation}\label{N_IBP}
N(\omega) = \frac{\omega^3}{(2\pi)^2 c^3} \int_{-\infty}^\infty g_\omega( \atan x ) 
 \frac{\mu(x)^2 x}{(1+x^2)^{3/2}}
 \left| \frac{\partial \mu}{\partial \omega} + \frac {\mu}{\omega}\right|
 d x .
\end{equation}

and the other derivation appears to get this correctly

\end{document}
